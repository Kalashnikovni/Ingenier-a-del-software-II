\nonstopmode
\documentclass[a4paper, 10pt]{article}
\usepackage{textcomp}
\usepackage[spanish]{babel}
\usepackage{amsmath}
\usepackage{amssymb}
\usepackage[margin=1in]{geometry}
\usepackage{indentfirst}
\usepackage{graphicx}
\usepackage[nounderscore]{syntax}

\setlength{\grammarindent}{1.5cm}

\author{
    Marzorati, Denise \\
}

\date{
    2017
}

\title{
    \Huge \textsc{{\bfseries R}esumen} \\
    \Huge \textsc{Patrones de diseño} \\
    \large \textsc{Ingeniería del software} \\
}

\begin{document}

\maketitle

\section*{Introducción}
  El diseño orientado a objetos suele ser difícil, pero es todavía más difícil hacer reusable dicho
diseño. Un buen diseño tiene que ser lo suficientemente específico como para resolver el problema,
pero lo suficientemente general como para resolver problemas futuros, y considerar nuevos
requerimientos.

\subsection*{¿Qué es un patrón de diseño?}
  Un patrón de diseño de compone de cuatro elementos básicos:

  \begin{itemize}
    \item {\bfseries Nombre}
    \item {\bfseries Problema}: describe cuándo aplicar el patrón. Explica el problema y su
contexto.
    \item {\bfseries Solución}: describe los elementos que conforman el diseño, sus relaciones,
responsabilidades y colaboraciones.
    \item{\bfseries Consecuencias}: son los resultados y las ventajas y desventajas obtenidas por
aplicar el patrón.
  \end{itemize}

  Entonces, un patrón de diseño serán descripciones acerca de objetos que se comunican entre sí, y
clases especialmente diseñados para resolver un problema de diseño general, en un contexto
particular.

\subsection*{Problemas que resuelven los patrones de diseño}

  \begin{itemize}
    \item {\bfseries Encontrar objetos apropiados}. Los patrones de diseño ayudan a identificar las
abstracciones poco explícitas, y los objetos que las capturan.
    \item{\bfseries Determinar la granularidad de un objeto}.
    \item{\bfseries Especificar la interfaz de los objetos}. Ayudan a definir interfaces
identificando los componentes claves de las mismas, y el tipo de datos que las subrutinas de ellas
reciben. Un patrón de diseño también puede indicar qué no poner en las interfaces. Además, también
definen relaciones entre las interfaces.
  \end{itemize}

\end{document}
